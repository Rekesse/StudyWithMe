\chapter{Sistemi di riferimento e massa invariante}
Negli eseperimenti di interazione fra particelle i sistemi di riferimento tipici sono due : quello del \textbf{laboratorio} e quello del \textbf{centro di massa}.
\section{Sistema del laboratorio}
Il sistema di riferimento del laboratorio è il sistema solidale con l'osservatore, si ha una particella bersaglio ferma e una particella che gli va incontro, si hanno 
quindi i due quadrimpulsi: 
\begin{align*}
        &\vb{P}_1 = \qty(\frac{E_1}{c},\va{p}_1) \\
        &\vb{P}_2 = \qty(\frac{E_2}{c}, 0) \\
        &\vb{P}_{tot} = \qty(\frac{E_1}{c} + m_2c, \va{p}_1) \tag*{essendo ferma $ \frac{E_2}{c} = m_2c $}
\end{align*}
\section{Sistema del centro di massa}
Il sistema del centro di massa è definito come quel sistema nel quale \textbf{l'impulso totale è nullo}.\\
Nel caso particolare in cui le due particelle hanno stessa massa ( es il collider ) il sistema del centro di massa conincide con il sistema del laboratorio. \\
\begin{tcolorbox}[colback=red!5!white,colframe=red!50!black,title=ATTENZIONE !]
        Indichiamo con $\vb{P}^{\prime}$ e $\va{p}^{\prime}$ rispettivamente i quadrimpulsi e i vettori impulso nel sistema di riferimento del centro di massa .
\end{tcolorbox}
\newpage
Abbiamo che : 
\begin{align*}
    &\vb{P}^{\prime}_1 = \qty(\frac{E^{\prime}_1}{c}, \va{p}^{\prime}) \\
    &\vb{P}^{\prime}_2 = \qty(\frac{E^{\prime}_2}{c}, -\va{p}^{\prime}) \\
    &\vb{P}^{\prime}_{tot} = \qty(\frac{E^{\prime}_1 +  E^{\prime}_2}{c}, 0) \end{align*}
\section{Massa invariante}
Consideriamo un sistema di N particelle ciascuna con il suo quadrimpulso $\vb{P}_{i} = \qty(\frac{E_{i}}{c},\va{p}_{i})$ come abbiamo detto 
possiamo definire il quadrimpulso totale $\vb{P}_{tot} = \sum_i \vb{P}_{i}$. \\
\begin{tcolorbox}[colback=red!5!white,colframe=red!50!black,title=ATTENZIONE !]
Ricordando che il modulo quadro di un quadrivettore è un invariante relativistico allora possiamo definire la \textbf{MASSA INVARIANTE} del sistema 
e indicheremo con $\sqrt{S}$ : 
\begin{align*}
    \sqrt{S} &= \sqrt{\vb{P}_{tot} \vdot \vb{P}_{tot}} \\
             &= \sqrt{\qty(\sum_i E_{i})^2 - \qty|\sum_i \va{p}_{i}|^2 }
\end{align*}
Come detto esso è un invariante quindi possiamo calcolarlo anche nel sistema del centro di massa $\sqrt{S^{\prime}}$
\begin{align*}
        \sqrt{S^{\prime}} = \sum_i E^{\prime}_{i} = E_{tot}
\end{align*}
Nota che nel centro di massa la massa invariante è denominata \textbf{Energia del centro di massa}
\end{tcolorbox}
\newpage
\subsection{Differenze fra lab e CdM : decadimento di una particella}
Consideriamo il decadimento del pione dato dalla seguente reazione : 
\begin{align*}
    \pi^{-} \rightarrow \mu^{-} + \nu_{\mu}
\end{align*}
conosciamo i valori delle masse, ci chiediamo quanta energia abbia il muone ( $m_{\nu} = 0$ )
\begin{center}\textbf{Lavoriamo nel CdM}\end{center}
Nelle reazioni la quantità importante da calcolare è $\sqrt{S}$ che è un invariante e si conserva, calcoliamocelo nel CdM e 
facciamolo prima nel sistema allo stato iniziale.\\
Inizialmente si ha una sola particella ossia il pione quindi il CdM conincide con il sistema di riferimento della particella stessa : 
\begin{align*}
        \vb{P}^{\prime}_{i} = \qty(m_{\pi}c^2,0) \rightarrow \sqrt{S^{\prime}_{i}} = m_{\pi}c^2
\end{align*}
Nello stato finale invece abbiamo due particelle ( scrivo direttamente il quadrimpulso totale ): 
\begin{align*}
        \vb{P}^{\prime}_{f} = \qty(\frac{E^{\prime}_{\mu} + E^{\prime}_{\nu}}{c},0) \rightarrow \sqrt{S^{\prime}_{f}} = E^{\prime}_{\mu} + E^{\prime}_{\nu} \tag*{ c = 1 }
\end{align*}
Per calcolare le energie bisogna trovare gli impulsi calcolati nel centro di massa, per farlo utilizziamo la \textbf{ conservazione dell'impulso }
\begin{align*}
        \va{p}_{\pi} &= \va{p}_{\mu} + \va{p}_{\nu} \\
        0 &= \textit{p}_{\mu} + \textit{p}_{\nu}\\
        \textit{p}_{\mu} &= - \textit{p}_{\nu}
\end{align*}
Possiamo ora calcolarci $\sqrt{S^{\prime}_{f}}$ eguagliarlo a $\sqrt{S^{\prime}_{i}}$ e trovarci così $\textit{p}_{\mu}$. \\
Tutto questo per dire che $\sqrt{S}$ e $E_{\mu}$ calcolati nel \textbf{centro di massa hanno valori ben definiti}. \\
\begin{center}\textbf{Lavoriamo nel Lab}\end{center}
Per trovarci $E_{\mu}$ nel Lab basta applicare le trasformazioni di Lorentz al valore che abbiamo trovato nel CdM : 
\begin{align*}
        E_{\mu} &= \gamma_{\pi}E^{\prime}_{\mu} + \beta_{\pi}\gamma_{\pi}\va{p}^{\prime}_{\mu} \\
                &= \gamma_{\pi}E^{\prime}_{\mu} + \beta_{\pi}\gamma_{\pi}\textit{p}^{\prime}_{\mu}\cos{\theta}\\
                &= \qty[\gamma_{\pi}E^{\prime}_{\mu} - \beta_{\pi}\gamma_{\pi}\textit{p}^{\prime}_{\mu} , \gamma_{\pi}E^{\prime}_{\mu} + \beta_{\pi}\gamma_{\pi}\textit{p}^{\prime}_{\mu}]  \tag*{$\cos{\theta} \in [-1,1]$} \\
                &\simeq \qty[105.6 MeV , 122.7 MeV]
\end{align*}
Perchè ciò avviene? \\
\begin{tcolorbox}[colback=red!5!white,colframe=red!50!black,title=ATTENZIONE !]
matematicamente ciò avviene poichè quando applichiamo le Trasformazioni di Lorentz la parte temporale ( energia ) si mescola con la parte spaziale 
quando cambio sistema di riferimento.
\end{tcolorbox}
