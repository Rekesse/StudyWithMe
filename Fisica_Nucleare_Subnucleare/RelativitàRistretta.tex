\documentclass[a4paper, 12pt, twoside]{report}

\usepackage{physics, amsmath}
\usepackage{enumitem}
\usepackage{hyperref}
\hypersetup{
        colorlinks=true,
                linkcolor=black,
                urlcolor=blue,
}

\usepackage{geometry}
\geometry{
        top=2cm,
                bottom=2cm,
                left=2cm,
                right=3cm,
                headheight=17pt,
                includeheadfoot,
}

\usepackage{fancyhdr, lastpage}
\pagestyle{fancy}
\fancyhf{}
\lhead{Mettere icona GitHub}
\rhead{ Relatività Ristretta }
\cfoot{Pagina \thepage\ di \pageref{LastPage}}
\renewcommand{\headrulewidth}{1.0pt}

\usepackage{etoolbox}
\patchcmd{\chapter}{\thispagestyle{plain}}{\thispagestyle{fancy}}{}{}

\title{Cinematica Relativistica}
\author{Pietro Garofalo}
\date{\today}


\begin{document}

\maketitle
\newpage
\tableofcontents

\chapter{Le trasformazioni di Lorentz }
In relatività le trasformazioni di Galileo sono sostituite dalle trasformazioni di Lorentz, prima di vederle nel dettaglio bisogna ricordarsi 
che le grandezze che ci interessano non sono più i semplici vettori ma i \textbf{ quadrivettori contravarianti } che definiamo nel seguente modo :
\begin{center}
        
        $ \vectorbold{X^{\mu}} = \begin{pmatrix} ct\\ \va{x} \end{pmatrix} $

\end{center}
Tale notazione evidenzia come i quadrivettori siano divisi in una parte temporale ( la prima componente ) e componenti spaziali ( vettore tridimensionale ), 
tali quaterne di valori trasformano, nel passaggio da un sistema di riferimento ad un altro, tramite le trasformazioni di Lorentz. \\
La metrica dei quadrivettori non è la metrica Euclidea bensì quella di \textbf{Minkowski}, se definiamo infatti due quadrivettori 
\begin{align*}
        \vectorbold{A^{\mu}} = \begin{pmatrix} a_0\\a_1\\a_2\\a_3\end{pmatrix}
        \
        \vectorbold{B^{\mu}} = \begin{pmatrix} b_0\\b_1\\b_2\\b_3\end{pmatrix}
\end{align*}
Allora il prodotto fra i due si definisce come :
\begin{align*}
        \vectorbold{A^{\mu}}\vdot\vectorbold{B_{\mu}} = a_0b_0 - a_1b_1 - a_2b_2 - a_3b_3
\end{align*}
dove $\vectorbold{B_{\mu}}$ non è altro che il \textbf{quadrivettore covariante} ossia il quadrivettore contravariante ma con il segno della parte spaziale opposto .\\

\end{document}

