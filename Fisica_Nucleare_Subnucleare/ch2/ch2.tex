\chapter{Il quadrivettore energia-impulso}
In meccanica relativistica non esiste il semplice vettore impulso ma esiste il vettore \textbf{quadrimpulso}, che ha come componenti energia e impulso 
e lo definiamo come : 
\begin{align*}
        \vb{P} \equiv p^{\mu} = \qty(\frac{E}{c}, \va{p}) \tag*{dove \ $ \va{p} = \begin{pmatrix} p_{x} \\ p_{y} \\ p_{z} \end{pmatrix}$} 
\end{align*}
In meccanica relativistica il quadrivettore impulso \textbf{NON} è un invariante relativistico, ma: 
\\
\begin{tcolorbox}[colback=red!5!white,colframe=red!50!black,title=ATTENZIONE !]
Il prodotto di un qualsiasi quadrivettore per se stesso è un invariante relativistico ossia che il suo valore non cambia se cambiamo sistema di riferimento . 
\end{tcolorbox}
Da ciò deduciamo che :
\begin{align*}
        \vb{P^{2}} = \vb{P^{\mu}}\vb{P_{\mu}} = \frac{E^{2}}{c^2} - \abs{\va{p}}^2
\end{align*}
Prima di andare avanti col calcolo di $\vb{P^{2}}$, indicheremo con $\textit{p} \equiv \abs{\va{p}}$, scriviamo qui delle relazioni fondamentali : \\
\begin{tcolorbox}[colback=red!5!white,colframe=red!50!black,title=ATTENZIONE !]
\begin{equation*}
        \left\{ \begin{aligned}
                        E &= m\gamma c^{2} \\
                        \va{p} &= m\gamma\va{v} 
                \end{aligned}
                \right.
\end{equation*}
\end{tcolorbox}
Allora si ottiene che il prodotto del quadrimpulso per se stesso è legato alla massa della particella : 
\begin{align*}
        \vb{P^2} = \frac{\qty(m\gamma c^{2})^2}{c^2} - \qty(m\gamma\textit{v})^2 = m^2c^2
\end{align*}
\newpage
Ponendo $c = 1$ e ricordando la forma iniziale di $\vb{P^2}$ si ottiene la seguente : \\
\begin{tcolorbox}[colback=red!5!white,colframe=red!50!black,title=ATTENZIONE !]
Alcune relazioni fondamentali : 
\begin{equation*}
        \left\{\begin{aligned}
        E^2 &= \textit{p}^2 + m^2\\
        \beta &=\frac{\textit{p}c}{E}\\ 
        \gamma &= \frac{E}{mc^2} \\
        \beta\gamma &= \frac{p}{mc}
\end{aligned}
\right.
\end{equation*}
Ricordiamo che : 
\begin{center} $\vb{P}$ e $E$ sono \textbf{grandezze conservate} \\ $\vb{P}^2$ è un invariante di Lorentz\end{center}
\end{tcolorbox}
Vediamo come si applicano le trasformazioni di Lorentz al quadrimpulso definendo sempre con l'apice $\prime$ le quantità 
riferite al sistema di riferimento in moto: 
\begin{equation*}
        \left\{\begin{aligned} 
                \frac{E^{\prime}}{c} &= \gamma\frac{E}{c} - \beta\gamma\textit{p}_{x} \\
                \textit{p}^{\prime}_{x} &= -\beta\gamma\frac{E}{c} + \gamma\textit{p}_{x}
        \end{aligned}
\right.
\end{equation*}
\section{Esercizi di riepilogo}
\subsection{Esercizio 1 : Energia e lavoro}
Quanto lavoro bisogna compiere per aumentare la velocità di un elettrone di massa \\ 
$m = 511 \frac{KeV}{c^2}$ dalla posizione a riposo a 0.50c ? 
\begin{center}{\textbf{Soluzione}}\end{center}
In generale il lavoro si calcola $ L = E_{f} - E_{i}$, ci sono due metodi :\\ 
\textbf{1) Forza bruta} \\
Questo metodo consiste nell'utilizzare la definizione di Energia : \\ 
\begin{align*}
        E_{f} &= \sqrt{m^2c^4 + \textit{p}^2c^2} \tag*{ricordando $\textit{p} = m\gamma\textit{v}$}\\ 
          &= \sqrt{m^2c^4 +m^2\gamma^2c^20.5^2c^2} \tag*{$\textit{v} = 0.5c$}\\
          &= 589 \ KeV
\end{align*}
\begin{tcolorbox}[colback=red!5!white,colframe=red!50!black,title=ATTENZIONE !]
Utile per i calcoli: \\
la massa se noti è espressa in $\frac{KeV}{c^2}$, questo vuol dire che quando abbiamo $mc^2$ 
vuol dire che si ha $\frac{KeV}{c^2}c^2 = KeV$
\end{tcolorbox}
Per l'energia iniziale si ha invece : \\
\begin{align*}
    E_{i} &= \sqrt{m^2c^4 + \textit{p}^2c^2} \\
          &= \sqrt{m^2c^4} \tag*{poichè $\textit{v}_{i} = 0$}\\
          &= mc^2 \\
          &= 511 \ KeV
\end{align*}
Si ottiene che lavoro $L = E_{f} - E_{i} = 589 KeV - 511 KeV = 78 KeV$ \\
\textbf{2) Usando le relazioni fondamentali}\\
Bisogna ricordarsi la relazione $E = m\gamma c^2$, infatti all'interno di gamma è contenuta la 
velocità, nel caso particolare $\textit{v} = 0$ allora $\gamma = 1$ : \\
\begin{align*}
        E_{f} &= m\gamma c^2 \tag*{$\gamma = \frac{1}{\sqrt{1-\frac{0.5c}{c}}}$}\\
        E_{i} &= mc^2\\
        L &= m\gamma c^2 - mc^2 = mc^2(\gamma - 1) = 78 KeV
\end{align*}
\subsection{Esercizio 2 : Decadimento e relatività speciale}
Un pione decade a riposo in un muone e un neutrino tramite il seguente processo : 
\begin{align*}
    \pi^{-} \rightarrow \mu^{-} + \nu_{\mu}
\end{align*}
Conoscendo i seguenti dati : $m_{\mu} = 105.6$Mev e il tempo proprio di vita $\tau_{\mu} = 2.2\mu s$ mentre il 
pione ha massa $m_{\pi} = 139.6$MeV.\\ 
Che distanza percorre il muone nel riferimento del laboratorio ? \\
\begin{center}{\textbf{Soluzione}}\end{center}
Innanzitutto quello che bisogna capire è che il tempo $\tau_{\mu}$ è il tempo proprio della particella ossia il tempo 
misurato nel sistema di riferimento della particella stessa, se volessimo calcolare L quindi che altro non è che la 
velocità per il tempo :\\
\begin{center} $L = \textit{v}t$\end{center}
Il tempo t ( il tempo nel sistema di riferimento del laboratorio ) non conincide con $\tau_{\mu}$ a causa della dilatazione dei tempi, 
si ha invece che $t=\gamma\tau_{\mu}$ ottenendo : 
\begin{align*}
        L &= c\beta\gamma\tau_{\mu}  \tag*{dove $\beta = \frac{\textit{v}}{c} \rightarrow \textit{v} = c\beta$} \\
          &= \frac{\textit{$p_{\mu}$}}{m_{\mu}}\tau_{\mu} \tag*{usata la relazione : $\beta\gamma = \frac{\textit{p}}{mc}$}
\end{align*}
Dove \textit{$p_{\mu}$} $\equiv \abs{\va{p_{\mu}}}$ che è quello che ci resta da calcolare, si hanno due metodi : \\
\textbf{1) Usiamo la legge della conservazione} \\
Sappiamo, già dalla meccanica classica che energia e impulso si conservano quindi possiamo scrivere la relazione prima del decadimento
e dopo ( nel sistema di riferimento del laboratorio ): 
\begin{align*}
        E_{\pi} &= E_{\mu} + E_{\nu} \\
        \va{p}_{\pi} &= \va{p}_{\mu} + \va{p}_{\nu}
\end{align*}
Vediamo ora le energie e impulsi per ogni particella : \\
\textbf{Pione:} \\
Sappiamo che è a riposo, otteniamo quindi che : 
\begin{equation*}
        \left\{ \begin{aligned}
                        \va{p}_{\pi} &= 0 \\
                        E_{\pi} &= \sqrt{m^2_{\pi}c^4 + \textit{p}_{\pi}^2c^2} = m_{\pi}c^2 
                \end{aligned}
                \right.
                \qquad \text{$E=mc^2$ è anche detta energia a riposo!}
\end{equation*}
\textbf{Muone:}\\
Del muone ci scriviamo solo l'energia siccome abbiamo già detto all'inizio che l'impulso è incognita : 
\begin{align*}
    E_{\mu} = \sqrt{m_{\mu}^2c^4 + \textit{p}_{\mu}^2c^2}
\end{align*}
\textbf{Neutrino:}\\
Avendo il neutrino massa nulla : 
\begin{align*}
    E_{\nu} = \textit{p}_{\nu}c
\end{align*}
\begin{tcolorbox}[colback=red!5!white,colframe=red!50!black,title=ATTENZIONE !]
        Dal fatto che $\va{p}_{\pi} = 0$ e per la conservazione otteniamo che : 
        \begin{align*}
                &0 = \va{p}_{\mu} + \va{p}_{\nu} \\
                &\va{p}_{\mu} = -\va{p}_{\nu} 
        \end{align*}
        Quello che ci serve per i calcoli è che : 
        \begin{align*}
            \abs{\va{p}_{\mu}} \equiv \textit{p}_{\mu} = \textit{p}_{\nu} \equiv \abs{\va{p}_{\nu}}
        \end{align*}
\end{tcolorbox}
Ossia il muone e il neutrino hanno stessa velocità e direzione ma verso opposto. \\
Possiamo ora continuare con i calcoli ripartendo dalla conservazione dell'energia: 
\begin{align*}
        &m_{\pi}c^2 = \sqrt{m_{\mu}^2c^4 + \textit{p}_{\mu}^2c^2} + \textit{p}_{\mu}c\\
        &m_{\pi}c^2 - \textit{p}_{\mu}c =\sqrt{m_{\mu}^2c^4 +\textit{p}_{\mu}^2c^2 }\\ 
        &m_{\pi}^2c^4 + \textit{p}_{\mu}^2c^2 - 2m_{\pi}\textit{p}_{\mu}c^3 = m_{\mu}^2c^4 +\textit{p}_{\mu}^2c^2
\end{align*}
da cui segue che : 
\begin{align*}
        \textit{p}_{\mu} = \frac{m_{\pi}^2 - m_{\mu}^2}{2m_{\pi}}c
\end{align*}
Allora la distanza che percorre, nel sistema di riferimento del laboratorio è : 
\begin{align*}
        L &= \frac{\textit{p}_{\mu}}{m_{\mu}}\tau_{\mu}\\
          &= \frac{m_{\pi}^2 - m_{\mu}^2}{2m_{\pi}}c\tau_{\mu}
\end{align*}
Passiamo ora al secondo metodo. \\
\textbf{2) Usando i quadrimpulsi}\\
La conservazione dell impulso vale anche per i quadrivettori, si può scrivere quindi : 
\begin{align*}
    \vb{P}_{\pi} = \vb{P}_{\mu} + \vb{P}_{\nu}
\end{align*}
Vogliamo ottenere $\textit{p}_{\mu}$ ossia il modulo della componente vettoriale del quadrimpulso, per farlo 
isoliamo $\vb{P}_{\mu}$ ottendendo : 
\begin{align*}
    \vb{P}_{\mu} = \vb{P}_{\pi} - \vb{P}_{\nu}
\end{align*}
Ricordando che \textbf{il prodotto scalare di un quadrivettore con se stesso è invariante} e che è uguale alla massa al riposo 
al quadrato : 
\begin{align*}
        m_{\mu}^2c^2 &= m_{\pi}^2c^2 + m_{\nu}^2c^2 -2\vb{P}_{\pi}\vdot\vb{P}_{\nu}  \tag*{$m_{\nu} = 0$} \\
        m_{\mu}^2c^2 &= m_{\pi}^2c^2 -2\qty(\frac{E_{\pi}}{c} \frac{E_{\nu}}{c} - \va{p}_{\pi}\vdot\va{p}_{\nu}) \tag*{$\va{p}_{\pi} = 0$ \ $E_{\nu} = \textit{p}_{\mu}c$}\\
        m_{\mu}^2c^2 &= m_{\pi}^2c^2 -2m_{\pi}\textit{p}_{\mu}c 
\end{align*}
Otteniamo la stessa forma di $\textbf{p}_{\mu}$ ottenuta col primo metodo . 
\newpage

